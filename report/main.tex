\documentclass[12pt]{article}

\usepackage{amsmath}
\usepackage{vntex}
\usepackage{titling}
\usepackage{graphicx}
\usepackage[table,xcdraw]{xcolor}

\title{
    \textbf{Automatic plant watering system}\\
}
\author{
    \textit{Author:}\\
    Phạm Đức Anh Khoa - 2053140\\
    Nguyễn Thị Ngọc Nhi - \\
    Mai Minh Nhật - \\
    Nguyễn Tôn Minh - \\~\\
    \textit{A report for Electrical - Electrical circuit subject}\\
    \textit{Faculty of}\\
    \textit{Computer Science and Engineering}
    }
\date{
    \includegraphics[scale = 0.6]{./images/Logo_BK.png}\\~\\
    \textbf{Viet Nam National University Ho Chi Minh} \\
    Ho Chi Minh University of Technology
    }

\begin{document}
    \maketitle
    \thispagestyle{empty}

    \newpage

    \section{Design}
        Vẽ mạch điện, tính toán công suất nên xài và module nguồn nào có sẵn trên thị trường phù hợp với mạch (trường hợp này xài 5W, 12V, 450 mA)

    \section{Experiments}
    \begin{table}[!h]
        \begin{tabular}{|
        >{\columncolor[HTML]{FFFFFF}}c |l|l|l|
        >{\columncolor[HTML]{FFCCC9}}l |}
        \hline
        \cellcolor[HTML]{FFCC67}\textbf{Order} & \multicolumn{1}{c|}{\cellcolor[HTML]{FFCC67}\textbf{Description}}                                                          & \multicolumn{1}{c|}{\cellcolor[HTML]{FFCC67}\textbf{Detail}}                                                                                                    & \multicolumn{1}{c|}{\cellcolor[HTML]{FFCC67}\textbf{Result}} & \multicolumn{1}{c|}{\cellcolor[HTML]{FFCC67}\textbf{Issue}}                                                                                   \\ \hline
        \textbf{1}                             & \begin{tabular}[c]{@{}l@{}}2 seperate \\ power supplies\end{tabular}                                                       & \begin{tabular}[c]{@{}l@{}}5W power supply for \\ Arduino UNO \\ and components\\ \\ 1W power supply \\ for water pump\end{tabular}                             & \cellcolor[HTML]{67FD9A}Good                                 & \begin{tabular}[c]{@{}l@{}}Have to plug in \\ 2 power supplies\end{tabular}                                                                   \\ \hline
        \textbf{2}                             & \begin{tabular}[c]{@{}l@{}}5V - 0.7A \\ power supply\end{tabular}                                                          & \begin{tabular}[c]{@{}l@{}}5V - 0.7A power \\ supply for the \\ whole circuit\end{tabular}                                                                      & \cellcolor[HTML]{FFCCC9}Unstable                             & \begin{tabular}[c]{@{}l@{}}Not enough power \\ causing components \\ to turn off randomly\end{tabular}                                        \\ \hline
        \textbf{3}                             & \begin{tabular}[c]{@{}l@{}}12V - 0.45A \\ power supply\\ \\ Same voltage \\ for all components\end{tabular}                & \begin{tabular}[c]{@{}l@{}}12V - 0.45A power \\ supply for the \\ whole circuit\\ \\ 5V for all components\end{tabular}                                         & \cellcolor[HTML]{FFCCC9}Unstable                             & \begin{tabular}[c]{@{}l@{}}Overheating\\ \\ Not enough power \\ causing Arduino \\ UNO to reset \\ when the pump \\ is activated\end{tabular} \\ \hline
        \textbf{4}                             & \begin{tabular}[c]{@{}l@{}}12V - 0.45A \\ power supply\\ \\ Different voltages \\ for different \\ components\end{tabular} & \begin{tabular}[c]{@{}l@{}}12V - 0.45A power \\ supply for the \\ whole circuit\\ \\ 5V for Arduino UNO \\ and components\\ \\ 3.3V for water pump\end{tabular} & \cellcolor[HTML]{67FD9A}Good                                 & \cellcolor[HTML]{67FD9A}No issue                                                                                                              \\ \hline
        \end{tabular}
    \end{table}

    \section{Implementing}
        \subsection{List of electronic devices}
            List linh kiện ra
        
        \subsection{Process}
            Quá trình làm lắp đặt: cố định trước, lắp dây sau, cách lắp dây (cắm thẳng vào mạch hay cắm vào bread board), loại dây mỏng hay dày (vì sao),...
        
        \subsection{Advantages and disadvantages}
            Ưu, nhược điểm của việc build bằng cách lắp dây vào bread board thay vì mạch, tại sao lại khogn6 in thẳng mạch ra, ...
    
    \section{Result}
        \subsection{Achievement}
        
        \subsection{eficiency}
\end{document}